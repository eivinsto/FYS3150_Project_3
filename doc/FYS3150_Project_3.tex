\documentclass[reprint,english,notitlepage]{revtex4-1}  % defines the basic parameters of the document

% if you want a single-column, remove reprint

% allows special characters (including æøå)
\usepackage[utf8]{inputenc}
\usepackage[english]{babel}

%% note that you may need to download some of these packages manually, it depends on your setup.
%% I recommend downloading TeXMaker, because it includes a large library of the most common packages.

\usepackage{physics,amssymb}  % mathematical symbols (physics imports amsmath)
\usepackage{graphicx}         % include graphics such as plots
\usepackage{xcolor}           % set colors
\usepackage{hyperref}         % automagic cross-referencing (this is GODLIKE)
\usepackage{tikz}             % draw figures manually
\usepackage{listings}         % display code
\usepackage{subfigure}        % imports a lot of cool and useful figure commands
\usepackage{cprotect}
\usepackage{float}

% defines the color of hyperref objects
% Blending two colors:  blue!80!black  =  80% blue and 20% black
\hypersetup{ % this is just my personal choice, feel free to change things
    colorlinks,
    linkcolor={red!50!black},
    citecolor={blue!50!black},
    urlcolor={blue!80!black}}

%% Defines the style of the programming listing
%% This is actually my personal template, go ahead and change stuff if you want
\lstnewenvironment{python}{
	\lstset{ %
		inputpath=,
		backgroundcolor=\color{white!88!black},
		basicstyle={\ttfamily\scriptsize},
		commentstyle=\color{magenta},
		language=Python,
		morekeywords={True,False},
		tabsize=4,
		stringstyle=\color{green!55!black},
		frame=single,
		keywordstyle=\color{blue},
		showstringspaces=false,
		columns=fullflexible,
		keepspaces=true}
}{}

\lstnewenvironment{cpp}{
	\lstset{ %
		inputpath=,
		backgroundcolor=\color{white!88!black},
		basicstyle={\ttfamily\scriptsize},
		commentstyle=\color{magenta},
		language=C++,
		morekeywords={True,False},
		tabsize=4,
		stringstyle=\color{green!55!black},
		frame=single,
		keywordstyle=\color{blue},
		showstringspaces=false,
		columns=fullflexible,
		keepspaces=true}
}{}

\lstset{literate=
  {á}{{\'a}}1 {é}{{\'e}}1 {í}{{\'i}}1 {ó}{{\'o}}1 {ú}{{\'u}}1
  {Á}{{\'A}}1 {É}{{\'E}}1 {Í}{{\'I}}1 {Ó}{{\'O}}1 {Ú}{{\'U}}1
  {à}{{\`a}}1 {è}{{\`e}}1 {ì}{{\`i}}1 {ò}{{\`o}}1 {ù}{{\`u}}1
  {À}{{\`A}}1 {È}{{\'E}}1 {Ì}{{\`I}}1 {Ò}{{\`O}}1 {Ù}{{\`U}}1
  {ä}{{\"a}}1 {ë}{{\"e}}1 {ï}{{\"i}}1 {ö}{{\"o}}1 {ü}{{\"u}}1
  {Ä}{{\"A}}1 {Ë}{{\"E}}1 {Ï}{{\"I}}1 {Ö}{{\"O}}1 {Ü}{{\"U}}1
  {â}{{\^a}}1 {ê}{{\^e}}1 {î}{{\^i}}1 {ô}{{\^o}}1 {û}{{\^u}}1
  {Â}{{\^A}}1 {Ê}{{\^E}}1 {Î}{{\^I}}1 {Ô}{{\^O}}1 {Û}{{\^U}}1
  {œ}{{\oe}}1 {Œ}{{\OE}}1 {æ}{{\ae}}1 {Æ}{{\AE}}1 {ß}{{\ss}}1
  {ű}{{\H{u}}}1 {Ű}{{\H{U}}}1 {ő}{{\H{o}}}1 {Ő}{{\H{O}}}1
  {ç}{{\c c}}1 {Ç}{{\c C}}1 {ø}{{\o}}1 {å}{{\r a}}1 {Å}{{\r A}}1
  {€}{{\euro}}1 {£}{{\pounds}}1 {«}{{\guillemotleft}}1
  {»}{{\guillemotright}}1 {ñ}{{\~n}}1 {Ñ}{{\~N}}1 {¿}{{?`}}1
}



\usepackage{thmtools}
\DeclareMathOperator{\nullspace}{Nul}
\DeclareMathOperator{\collspace}{Col}
\DeclareMathOperator{\rref}{Rref}
%%\DeclareMathOperator{\dim}{Dim}

 % "meq": must be equal
\newcommand{\meq}{\overset{!}{=}}
\newcommand\numberthis{\addtocounter{equation}{1}\tag{\theequation}}

\newcommand{\R}{\mathbb{R}}
\newcommand*\Heq{\ensuremath{\overset{\kern2pt L'H}{=}}}
\usepackage{bm}
\newcommand{\uveci}{{\bm{\hat{\textnormal{\bfseries\i}}}}}
\newcommand{\uvecj}{{\bm{\hat{\textnormal{\bfseries\j}}}}}
\DeclareRobustCommand{\uvec}[1]{{%
  \ifcsname uvec#1\endcsname
     \csname uvec#1\endcsname
   \else
    \bm{\hat{\mathbf{#1}}}%
   \fi
}}
\usepackage[binary-units=true]{siunitx}

\makeatletter
\newcommand*{\balancecolsandclearpage}{%
  \close@column@grid
  \cleardoublepage
  \twocolumngrid
}
\makeatother

\newcounter{subproject}
\renewcommand{\thesubproject}{\alph{subproject}}
\newenvironment{subproj}{
\begin{description}
	\item[\refstepcounter{subproject}(\thesubproject)]
}{\end{description}}


\begin{document}
\title{Project 3}   % self-explanatory
\author{Eivind Støland, Anders P. Åsbø}               % self-explanatory
\date{\today}                             % self-explanatory
\noaffiliation                            % ignore this

\maketitle                                % creates the title, author, date


\tableofcontents

\section{Introduction} \label{sec:I}

\section{Formalism} \label{sec:II}

\subsection{A generalized gravitational force}
For any two bodies \(a\) and \(b\), the gravitational force acting on body \(a\) from body \(b\) is
\begin{align*}
	\vec{F}_{a} = -G\frac{m_{a}m_{b}}{|\vec{r}_{a}-\vec{r}_{b}|^{3}}\left(\vec{r}_{a}-\vec{r}_{b}\right), \numberthis \label{eq:grav} \\
\end{align*}
where \(m_{a}\) and \(m_{b}\) are the masses of \(a\) and \(b\) respectivly, \(\vec{r}_{a}\) and \(\vec{r}_{b}\) are the position vectors of \(a\) and \(b\) respectivly, and \(G\) is the gravitational constant.

If we consider a body in a \(n\)-body system where the only interaction is gravitational, the net force on any body \(i\) becomes the sum of all the forces on body \(i\) from every other body in the system. Using \eqref{eq:grav} and summing over all bodies \(j\), where \(j\neq i\), we get
\begin{align*}
	\vec{F}_{i} = -Gm_{i}\sum_{j\neq i}m_{j}\frac{\vec{r}_{i}-\vec{r}_{j}}{|\vec{r}_{i}-\vec{r}_{j}|^{3}}. \numberthis \label{eq:gengrav}
\end{align*}

\section{Method} \label{sec:III}

\section{Results} \label{sec:IV}

\section{Discussion} \label{sec:V}

\section{Conclusion} \label{sec:VI}
\onecolumngrid
\bibliography{kilder.bib}{}
\newpage
\twocolumngrid

\appendix
\section{Source code} \label{A}
All code for this report was written in C++ and Python 3.8, and the complete set of files can be found at
\url{https://github.com/eivinsto/FYS3150_Project2.git}.

\cprotect\subsection{\verb+project.py+} \label{A.1}
Main script for running project, plotting and data analysis.

\url{https://github.com/eivinsto/FYS3150_Project_3/blob/main/project.py}

\cprotect\subsection{\verb+main.cpp+} \label{A.2}
Main cpp file for running algorithms.

\url{https://github.com/eivinsto/FYS3150_Project_3/blob/main/src/main.cpp}

\cprotect\subsection{\verb+solar_integrator+} \label{A.3}
Source and header file containing class for integrating a velocity and position for a system of bodies.

Source file \verb+solar_integrator.cpp+:
\url{https://github.com/eivinsto/FYS3150_Project_3/blob/main/src/solar_integrator.cpp}

Header file \verb+solar_integrator.hpp+:
\url{https://github.com/eivinsto/FYS3150_Project_3/blob/main/src/solar_integrator.hpp}

\cprotect\subsection{\verb+solar_system+} \label{A.4}
Source and header file containing class for creating system of bodies, and calculating gravitational force between them, as well as kinetic and gravitational-potential energy of system.

Source file \verb+solar_system.cpp+:
\url{https://github.com/eivinsto/FYS3150_Project_3/blob/main/src/solar_system.cpp}

Header file \verb+solar_system.hpp+:
\url{https://github.com/eivinsto/FYS3150_Project_3/blob/main/src/solar_system.hpp}

\cprotect\subsection{\verb+celestila_body+} \label{A.5}
Source and header file containing class defining gravitational bodies.

Source file \verb+celestial_body.cpp+:
\url{https://github.com/eivinsto/FYS3150_Project_3/blob/main/src/celestial_body.cpp}

Header file \verb+celestial_body.hpp+:
\url{https://github.com/eivinsto/FYS3150_Project_3/blob/main/src/celestial_body.hpp}

\cprotect\subsection{\verb+test_main.cpp+} \label{A.6}
File running unit-tests using CATCH2 framework.

\url{https://github.com/eivinsto/FYS3150_Project_3/blob/main/src/test_main.cpp}

\cprotect\subsection{\verb+test_functions.cpp+} \label{A.7}
File containing the unit-tests for the system.

\url{https://github.com/eivinsto/FYS3150_Project_3/blob/master/src/test_functions.cpp}

\newpage
\section{Selected results} \label{B}
Here is a folder of selected results from running our code.

\url{https://github.com/eivinsto/FYS3150_Project_3/tree/master/data}

\newpage
\section{System specifications} \label{C}
All results included in this report were achieved by running the implementation on the following system:
\begin{itemize}
	\item CPU: AMD Ryzen \(9\) \(3900\)X
	\item RAM: \(2\times\SI{8}{\giga\byte}\) Corsair Vengeance LPX DDR\(4\) \(\SI{3200}{\mega\hertz}\)
\end{itemize}

\end{document}