\documentclass[reprint,english,notitlepage]{revtex4-1}  % defines the basic parameters of the document

% if you want a single-column, remove reprint

% allows special characters (including æøå)
\usepackage[utf8]{inputenc}
\usepackage[english]{babel}

%% note that you may need to download some of these packages manually, it depends on your setup.
%% I recommend downloading TeXMaker, because it includes a large library of the most common packages.

\usepackage{physics,amssymb}  % mathematical symbols (physics imports amsmath)
\usepackage{graphicx}         % include graphics such as plots
\usepackage{xcolor}           % set colors
\usepackage{hyperref}         % automagic cross-referencing (this is GODLIKE)
\usepackage{tikz}             % draw figures manually
\usepackage{listings}         % display code
\usepackage{subfigure}        % imports a lot of cool and useful figure commands
\usepackage{cprotect}
\usepackage{float}

% defines the color of hyperref objects
% Blending two colors:  blue!80!black  =  80% blue and 20% black
\hypersetup{ % this is just my personal choice, feel free to change things
    colorlinks,
    linkcolor={red!50!black},
    citecolor={blue!50!black},
    urlcolor={blue!80!black}}

%% Defines the style of the programming listing
%% This is actually my personal template, go ahead and change stuff if you want
\lstnewenvironment{python}{
	\lstset{ %
		inputpath=,
		backgroundcolor=\color{white!88!black},
		basicstyle={\ttfamily\scriptsize},
		commentstyle=\color{magenta},
		language=Python,
		morekeywords={True,False},
		tabsize=4,
		stringstyle=\color{green!55!black},
		frame=single,
		keywordstyle=\color{blue},
		showstringspaces=false,
		columns=fullflexible,
		keepspaces=true}
}{}

\lstnewenvironment{cpp}{
	\lstset{ %
		inputpath=,
		backgroundcolor=\color{white!88!black},
		basicstyle={\ttfamily\scriptsize},
		commentstyle=\color{magenta},
		language=C++,
		morekeywords={True,False},
		tabsize=4,
		stringstyle=\color{green!55!black},
		frame=single,
		keywordstyle=\color{blue},
		showstringspaces=false,
		columns=fullflexible,
		keepspaces=true}
}{}

\lstset{literate=
  {á}{{\'a}}1 {é}{{\'e}}1 {í}{{\'i}}1 {ó}{{\'o}}1 {ú}{{\'u}}1
  {Á}{{\'A}}1 {É}{{\'E}}1 {Í}{{\'I}}1 {Ó}{{\'O}}1 {Ú}{{\'U}}1
  {à}{{\`a}}1 {è}{{\`e}}1 {ì}{{\`i}}1 {ò}{{\`o}}1 {ù}{{\`u}}1
  {À}{{\`A}}1 {È}{{\'E}}1 {Ì}{{\`I}}1 {Ò}{{\`O}}1 {Ù}{{\`U}}1
  {ä}{{\"a}}1 {ë}{{\"e}}1 {ï}{{\"i}}1 {ö}{{\"o}}1 {ü}{{\"u}}1
  {Ä}{{\"A}}1 {Ë}{{\"E}}1 {Ï}{{\"I}}1 {Ö}{{\"O}}1 {Ü}{{\"U}}1
  {â}{{\^a}}1 {ê}{{\^e}}1 {î}{{\^i}}1 {ô}{{\^o}}1 {û}{{\^u}}1
  {Â}{{\^A}}1 {Ê}{{\^E}}1 {Î}{{\^I}}1 {Ô}{{\^O}}1 {Û}{{\^U}}1
  {œ}{{\oe}}1 {Œ}{{\OE}}1 {æ}{{\ae}}1 {Æ}{{\AE}}1 {ß}{{\ss}}1
  {ű}{{\H{u}}}1 {Ű}{{\H{U}}}1 {ő}{{\H{o}}}1 {Ő}{{\H{O}}}1
  {ç}{{\c c}}1 {Ç}{{\c C}}1 {ø}{{\o}}1 {å}{{\r a}}1 {Å}{{\r A}}1
  {€}{{\euro}}1 {£}{{\pounds}}1 {«}{{\guillemotleft}}1
  {»}{{\guillemotright}}1 {ñ}{{\~n}}1 {Ñ}{{\~N}}1 {¿}{{?`}}1
}



\usepackage{thmtools}
\DeclareMathOperator{\nullspace}{Nul}
\DeclareMathOperator{\collspace}{Col}
\DeclareMathOperator{\rref}{Rref}
%%\DeclareMathOperator{\dim}{Dim}

 % "meq": must be equal
\newcommand{\meq}{\overset{!}{=}}
\newcommand\numberthis{\addtocounter{equation}{1}\tag{\theequation}}

\newcommand{\R}{\mathbb{R}}
\newcommand*\Heq{\ensuremath{\overset{\kern2pt L'H}{=}}}
\usepackage{bm}
\newcommand{\uveci}{{\bm{\hat{\textnormal{\bfseries\i}}}}}
\newcommand{\uvecj}{{\bm{\hat{\textnormal{\bfseries\j}}}}}
\DeclareRobustCommand{\uvec}[1]{{%
  \ifcsname uvec#1\endcsname
     \csname uvec#1\endcsname
   \else
    \bm{\hat{\mathbf{#1}}}%
   \fi
}}
\usepackage[binary-units=true]{siunitx}

\makeatletter
\newcommand*{\balancecolsandclearpage}{%
  \close@column@grid
  \cleardoublepage
  \twocolumngrid
}
\makeatother

\newcounter{subproject}
\renewcommand{\thesubproject}{\alph{subproject}}
\newenvironment{subproj}{
\begin{description}
	\item[\refstepcounter{subproject}(\thesubproject)]
}{\end{description}}


\begin{document}
\title{Project 3}   % self-explanatory
\author{Eivind Støland, Anders P. Åsbø}               % self-explanatory
\date{\today}                             % self-explanatory
\noaffiliation                            % ignore this

\maketitle                                % creates the title, author, date


\tableofcontents

\section{Introduction} \label{sec:I}

\section{Formalism} \label{sec:II}

\subsection{Two methods for numerical integration}
To solve our equations of motion we require methods for numerical integration. To derive our first method, we consider the generic function \(f\) that depends on a variable \(t\) such that
\begin{align*}
	\frac{d}{dt}f(t) = g(t, f(t)). \numberthis \label{eq:fofg}
\end{align*}
Given an arbitrary point \(f(t_{0})\), we wish to approximate the point \(f(t_{1})\). Thus we can integrate each side of \eqref{eq:fofg}, such that
\begin{align*}
	\int_{t_{0}}^{t_{1}} \frac{d}{dt} f(t) dt = \int_{t_{0}}^{t_{1}} g(t, f(t)) dt.
\end{align*}
Applying the fundemental theorem of calculus, we get that
\begin{align*}
	f(t_{1}) - f(t_{0}) = \int_{t_{0}}^{t_{1}} g(t, f(t)) dt, \\
	f(t_{1}) = f(t_{0}) + \int_{t_{0}}^{t_{1}} g(t, f(t)) dt. \numberthis \label{eq:foft1}
\end{align*}
To complete the approximation, we define a step-length \(\Delta t = t_{1}-t_{0}\), and approximate the area under the curve of \(g(t, f(t))\) between \(g(t_{0}, f(t_{0})\) and \(g(t_{1}, f(t_{1}))\) as the area of a rectangle whith height \(g(t_{0}, f(t_{0}))\). Inserting the rectangle approximation into \eqref{eq:foft1} yields
\begin{align*}
	f(t_{0} + \Delta t) \approx f(t_{0}) + \Delta t g(t_{0}, f(t_{0})).
\end{align*}

As we neglect all terms of second order or higher in this approximation we have that the error goes as $\mathcal{O}(\Delta t^2)$. 
If we want to approximate a value \(f(t_{k+1})\), we can simply start with a known value at a \(t_{k} < t_{k+1}\) and take a step with our approximation such that
\begin{align*}
	f(t_{k+1}) \approx f(t_{k}) + \Delta t g(t_{k}, f(t_{k})).
\end{align*}
Finally we can simplify the notation by setting \(f_{k+1} = f(t_{k})\) resulting in
\begin{align*}
	f_{k+1} = f_{k} + \Delta t g(t_{k}, f_{k}). \numberthis \label{eq:forwardEuler}
\end{align*}
This method of numerical integration is known as Euler's method, henceforth referred to as the forward Euler method.

Since we are approximating the integral between each pair of \(g(t, f_{k})\) and \(g(t, f_{k+1})\)  as a single rectangle with width \(\Delta t\), our approximation will be more accurate for smaller values of \(\Delta t\). Thus to approximate the curve of the function \(f\), we create a set of \(f_{k}\) values for a given \(f_{0}\) and \(\Delta t\) by repeatedly taking steps using the forward Euler method.

Many physical problems can be represented by a second order differential equation:

\begin{align*}
\frac{d^2}{dt^2} \vec{x}(t) &= \vec{a}(t,\vec{x}(t)) \, , \numberthis \label{eq:sodiff}
\end{align*}

where $\vec{x}(t)$ is the function we want to find and $\vec{a}(t,\vec{x}(t))$ is a function dependent on $\vec{x}(t)$. As the parts of this equation are vectors, this is actually a set of $D$ differential equations, where $D$ is the length of the vectors. This differential equation can be split into $2D$ first order differential equations:

\begin{align*}
\frac{d}{dt} \vec{v}(t) &= \vec{a}(t,\vec{x}(t))  \\
\frac{d}{dt} \vec{x}(t) &= \vec{v}(t) \, , \numberthis \label{eq:sodiff_split}
\end{align*}

When split into first order differential equations like this we can easily solve this system using methods such as the forward Euler method given a set of initial conditions. 

%In this kind of system we can make a small adaptation to the forward Euler method. The method with this small change is called the Euler-Cromer method (using the same discretization that was described earlier): 

%\begin{align*}
%\vec{v}_{k+1} &= \vec{v}_k + \Delta t  \vec{a}(t_k, \vec{x}_k) \\
%\vec{x}_{k+1} &= \vec{x}_k + \Delta t \vec{v}_{k+1} \numberthis \label{eq:EulerCromer}
%\end{align*}  

%Instead of using $\vec{v}_k$ to calculate $\vec{x}_{k+1}$ we use the next step $\vec{v}_{k+1}$. This improves the accuracy of the numerical solution. 

Another method used to solve the system of equations in \eqref{eq:sodiff_split} is the Velocity Verlet algorithm. We will now derive this algorithm. First we Taylor expand $\vec{x}(t + \Delta t)$:

\begin{align*}
\vec{x}(t + \Delta t) &= \vec{x}(t) + \Delta t \vec{v}(t) + \frac{\Delta t^2}{2} \vec{a}(t) + \mathcal{O}(\Delta t^3) \numberthis \label{eq:so_taylorx}
\end{align*}

Using the discretization described earlier, along with defining $\vec{x}(t_k) = \vec{x}_k$ and $\vec{v}(t_k) = \vec{v}_k$ we can rewrite this equation: 

\begin{align*}
\vec{x}_{k + 1} &\approx \vec{x}_k + \Delta t \vec{v}_k + \frac{\Delta t^2}{2} \vec{a}(t_k,\vec{x}_k) \numberthis \label{eq:so_taylorx_disc}
\end{align*}

Similarly we can also Taylor expand $\vec{v}(t+\Delta t)$:

\begin{align*}
\vec{v}(t+\Delta t) &= \vec{v}(t) + \Delta t \vec{a}(t) + \frac{\Delta t^2}{2} \frac{d}{dt} \vec{a}(t) + \mathcal{O}(\Delta t^3) \numberthis \label{eq:so_taylorv}
\end{align*}

We do not know what $\frac{d}{dt} \vec{a}(t)$ is, but we can approximate it:

\begin{align*}
\frac{d}{dt}\vec{a}(t) &= \frac{\vec{a}(t+\Delta t) - \vec{a}(t)}{\Delta t} + \mathcal{O}(\Delta t^2) \\
\Delta t \frac{d}{dt} \vec{a}(t) &= \vec{a}(t+ \Delta t ) - \vec{a}(t) + \mathcal{O}(\Delta t^3) 
\end{align*}

Inserting this into \eqref{eq:so_taylorv} gives:

\begin{align*}
\vec{v}(t+\Delta t) &= \vec{v}(t) + \Delta t \vec{a}(t) + \frac{\Delta t}{2} (\vec{a}(t+ \Delta t ) - \vec{a}(t)) + \mathcal{O}(\Delta t^3) \\
&= \vec{v}(t) + \frac{\Delta t}{2} (\vec{a}(t+ \Delta t ) + \vec{a}(t)) +  \mathcal{O}(\Delta t^3) 
\end{align*}

Discretizing this gives:

\begin{align*}
\vec{v}_{k+1} &\approx \vec{v}_k + \frac{\Delta t}{2} \bigg( \vec{a}(t_{k+1}, \vec{x}_{k+1}) + \vec{a}(t_k, \vec{x}_k) \bigg)
\end{align*}

Combining this with \eqref{eq:so_taylorx_disc} gives us the Velocity Verlet algorithm:

\begin{align*}
\vec{x}_{k+1} &= \vec{x}_k + \Delta t \vec{v}_k + \frac{1}{2} \Delta t^2 \vec{a}(t_k,\vec{x}_k) \\
\vec{v}_{k+1} &= \vec{v}_k + \frac{1}{2}\Delta t \bigg( \vec{a}(t_{k+1},x_{k+1}) + \vec{a}(t_k, \vec{x}_k) \bigg) \, . \numberthis \label{eq:VelocityVerlet}
\end{align*}

In this algorithm we first move the position one step, and then use this when calculating $\vec{v}$ in the next step. This algorithm has two advantages over the forward Euler method. Firstly it is approximately symplectic, meaning that if it is used to solve a set of equations given by Newton's second law there is little to no energy drift. Secondly the error of the approximation goes as $\mathcal{O}(\Delta t^3)$ compared to $\mathcal{O}(\Delta t^2)$ for the forward Euler method. 

\subsection{A generalized gravitational force}
For any two bodies \(a\) and \(b\), the gravitational force acting on body \(a\) from body \(b\) is
\begin{align*}
	\vec{F}_{a} = -G\frac{m_{a}m_{b}}{|\vec{r}_{a}-\vec{r}_{b}|^{3}}\left(\vec{r}_{a}-\vec{r}_{b}\right), \numberthis \label{eq:grav} \\
\end{align*}
where \(m_{a}\) and \(m_{b}\) are the masses of \(a\) and \(b\) respectivly, \(\vec{r}_{a}\) and \(\vec{r}_{b}\) are the position vectors of \(a\) and \(b\) respectivly, and \(G\) is the gravitational constant.

If we consider a body in a \(n\)-body system where the only interaction is gravitational, the net force on any body \(i\) becomes the sum of all the forces on body \(i\) from every other body in the system. Using \eqref{eq:grav} and summing over all bodies \(j\), where \(j\neq i\), we get
\begin{align*}
	\vec{F}_{i} = -Gm_{i}\sum_{j\neq i}m_{j}\frac{\vec{r}_{i}-\vec{r}_{j}}{|\vec{r}_{i}-\vec{r}_{j}|^{3}}. \numberthis \label{eq:gengrav}
\end{align*}

We then find the acceleration of body \(i\) at time \(t\) by dividing the net force by \(m_{a}\) resulting in
\begin{align*}
	\vec{a}_{i}(t) = \frac{\vec{F}_{i}(t)}{m_{i}} = -G\sum_{j\neq i}m_{j}\frac{\vec{r}_{i}(t)-\vec{r}_{j}(t)}{|\vec{r}_{i}(t)-\vec{r}_{j}(t)|^{3}},
\end{align*}
which can be written as the differential equation
\begin{align*}
 	\frac{d\vec{v}_{i}(t)}{dt} = -G\sum_{j\neq i}m_{j}\frac{\vec{r}_{i}(t)-\vec{r}_{j}(t)}{|\vec{r}_{i}(t)-\vec{r}_{j}(t)|^{3}}. \numberthis \label{eq:gendv}
\end{align*}
In addition we have the differential equation
\begin{align*}
	\frac{d\vec{r}_{i}(t)}{dt} = \vec{v}(t) \numberthis \label{eq:gendr}.
\end{align*}

Solving \eqref{eq:gendv} and \eqref{eq:gendr} for every body \(i=1, ..., n\), allows us to find the internal motion of a given n-body system with only gravitational interaction.

\section{Method} \label{sec:III}

\section{Results} \label{sec:IV}

\section{Discussion} \label{sec:V}

\section{Conclusion} \label{sec:VI}
\onecolumngrid
\bibliography{kilder.bib}{}
\newpage
\twocolumngrid

\appendix
\section{Source code} \label{A}
All code for this report was written in C++ and Python 3.8, and the complete set of files can be found at
\url{https://github.com/eivinsto/FYS3150_Project2.git}.

\cprotect\subsection{\verb+project.py+} \label{A.1}
Main script for running project, plotting and data analysis.

\url{https://github.com/eivinsto/FYS3150_Project_3/blob/main/project.py}

\cprotect\subsection{\verb+main.cpp+} \label{A.2}
Main cpp file for running algorithms.

\url{https://github.com/eivinsto/FYS3150_Project_3/blob/main/src/main.cpp}

\cprotect\subsection{\verb+solar_integrator+} \label{A.3}
Source and header file containing class for integrating a velocity and position for a system of bodies.

Source file \verb+solar_integrator.cpp+:
\url{https://github.com/eivinsto/FYS3150_Project_3/blob/main/src/solar_integrator.cpp}

Header file \verb+solar_integrator.hpp+:
\url{https://github.com/eivinsto/FYS3150_Project_3/blob/main/src/solar_integrator.hpp}

\cprotect\subsection{\verb+solar_system+} \label{A.4}
Source and header file containing class for creating system of bodies, and calculating gravitational force between them, as well as kinetic and gravitational-potential energy of system.

Source file \verb+solar_system.cpp+:
\url{https://github.com/eivinsto/FYS3150_Project_3/blob/main/src/solar_system.cpp}

Header file \verb+solar_system.hpp+:
\url{https://github.com/eivinsto/FYS3150_Project_3/blob/main/src/solar_system.hpp}

\cprotect\subsection{\verb+celestila_body+} \label{A.5}
Source and header file containing class defining gravitational bodies.

Source file \verb+celestial_body.cpp+:
\url{https://github.com/eivinsto/FYS3150_Project_3/blob/main/src/celestial_body.cpp}

Header file \verb+celestial_body.hpp+:
\url{https://github.com/eivinsto/FYS3150_Project_3/blob/main/src/celestial_body.hpp}

\cprotect\subsection{\verb+test_main.cpp+} \label{A.6}
File running unit-tests using CATCH2 framework.

\url{https://github.com/eivinsto/FYS3150_Project_3/blob/main/src/test_main.cpp}

\cprotect\subsection{\verb+test_functions.cpp+} \label{A.7}
File containing the unit-tests for the system.

\url{https://github.com/eivinsto/FYS3150_Project_3/blob/master/src/test_functions.cpp}

\newpage
\section{Selected results} \label{B}
Here is a folder of selected results from running our code.

\url{https://github.com/eivinsto/FYS3150_Project_3/tree/master/data}

\newpage
\section{System specifications} \label{C}
All results included in this report were achieved by running the implementation on the following system:
\begin{itemize}
	\item CPU: AMD Ryzen \(9\) \(3900\)X
	\item RAM: \(2\times\SI{8}{\giga\byte}\) Corsair Vengeance LPX DDR\(4\) \(\SI{3200}{\mega\hertz}\)
\end{itemize}

\end{document}